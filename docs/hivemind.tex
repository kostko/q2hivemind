\documentclass[a4paper,10pt]{article}
\usepackage[utf8]{inputenc}
\usepackage[slovene]{babel}
\usepackage[left=2cm,top=2.3cm,right=2cm,nohead]{geometry}
\usepackage{amsmath}
\usepackage{amsfonts}
\usepackage{amssymb}
\usepackage[pdftex]{graphicx}
\usepackage{subfig}
\usepackage{textcomp}
\usepackage{tikz}

\begin{document}
\begin{titlepage}
  \let\footnotesize\small
  \let\footnoterule\relax
  \let \footnote \thanks
  \null\vfil
  \vskip 60pt
  \begin{center}
    {\LARGE \textbf{Hivemind} \\ Porazdeljeni Quake 2 bot \par}
    \vskip 3em
    {\large
     \lineskip .75em
      \begin{tabular}[t]{c}
        Grega Kešpret \\
        Jernej Kos \\
        Anže Vavpetič
      \end{tabular}\par}
      \vskip 1.5em
    {\large \today \par}
  \end{center}\par
  \vfil\null
\end{titlepage}


\section{Predstavitev in razdelitev problema}

\section{Pregled arhitekture rešitve}

\section{Vmesnik do virtualnega sveta}

\section{Nadzor gibanja in izogibanje oviram}

\subsection{Senzorji}

\subsection{Nadzorni sistem z EANN}

\subsection{Nadzorni sistem z mehko logiko}

\section{Navigacija po svetu}

\subsection{Statična geometrija iz BSP map}

\subsection{Samodejno učenje topografije}

\subsection{Iskanje poti}

\section{Stanja robota}

\subsection{Arbitraža}

\subsection{Reinforcement learning}
% TODO a obstaja kakšen boljši slovenski izraz za RL?

\section{Komunikacija med instancami}

\subsection{MOLD (Message Oriented Lightweight Distributor)}

\subsection{Potek komunikacije}

\section{Možne izboljšave}

\section{Zaključek}

\end{document}

